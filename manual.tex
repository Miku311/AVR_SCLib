\documentclass{article}
\usepackage{ctex}
\usepackage{listings} 
\title{AVR\_SCLib manual}
\usepackage[colorlinks,
linkcolor=black,
anchorcolor=blue,
citecolor=green
]{hyperref}
\begin{document}
\author{Schummacher S.J.F}
\maketitle
\newpage
\tableofcontents
\newpage
\section{综述}
作者本未想做一套轮子,但使用中深感不便,每次实现一些重复性强的功能都要重新写一遍,实在是不胜其烦就打算写一些自己用且简单粗暴的函数,但写着写着函数也就慢慢多了起来,渐渐的整理出来一些库,现在把他开源出来,希望用AVR的人可以少造一些轮子。\par
至于为什么不用默认的README.md和Wiki嘛,我只能坦白的说我不会Markdown和Wiki只会\LaTeX
\section{ADC}
\subsection{ADC\_Date}
\begin{lstlisting}[language=C]
unsigned int ADC_Date(unsigned char i);
\end{lstlisting}
其中参量i是PA组的IO口数,返回左对齐的10bit采样
\subsection{eg.}
我们要采PA1的样,并将ADC采样值付给变量ADC\_temp:
\begin{lstlisting}[language=C]
ADC_temp =  ADC_Date(1);
\end{lstlisting}
\section{Devich}
Device.h里面的文件有点类似于很多人写的main.h但是我在这里还增加了一些单片机常用的一些宏定义
\begin{lstlisting}[language=C]
#define LSL(x, y) x=x<<y
#define LSR(x, y) x=x>>y
\end{lstlisting}
把x左/右移y位并将结果付给x
\begin{lstlisting}[language=C]
#define MAX(a, b) a>b?a:b
#define MIN(a, b) a<b?a:b
\end{lstlisting}
返回a, b中最大/小的数
\begin{lstlisting}[language=C]
#define SEI asm("sei");
#define CLI asm("cli");
\end{lstlisting}
全局中断使能/清除
\section{EEPROM}
\subsection{EEPROM\_write}
\begin{lstlisting}[language=C]
void EEPROM_write(unsigned int uiAddress, unsigned char ucData);
\end{lstlisting}
uiAddress是数据需要储存的地址,ucData是数据
\subsection{EEPROM\_write}
\begin{lstlisting}[language=C]
unsigned char EEPROM_read(unsigned int uiAddress);
\end{lstlisting}
uiAddress是数据需要读出的地址,返回值是读出的数据
\subsection{eg.}
我们将变量i储存到地址0x00,然后将地址为0x01的数据读出到j
\begin{lstlisting}[language=C]
EEPROM_write(0x00, i);
j = EEPROM_read(0x01);
\end{lstlisting}

\end{document}